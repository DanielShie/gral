  \xname{CombinatorialLayer}
  \begin{Label}{CombinatorialLayer}
    \introsection{The combinatorial grid layer}
  \end{Label}
  Central to the notion of a grid are the combinatorial relationships
  among its entities of different dimension, namely, their incidence relation.

  The combinatorial grid layer consists of 
  (combinatorial) {\em Grids\/} \sectionlink{(C)}{Grid}
  as comprehensive entities,
  {\em grid elements\/} \sectionlink{(C)}{GridElement},
  like Vertex, Edge, or Cell, which are its building blocks,
  {\em element handles\/} \sectionlink{(C)}{GridElementHandle}
  (minimal representations or indices of elements),
  {\em sequences iterators\/}  \sectionlink{(C)}{GridSequenceIterator}
  that allow to iterate over all elements of a given type,
  and {\em incidence iterators\/} \sectionlink{(C)}{GridIncidenceIterator}
  allowing to iterate over all elements of some fixed dimension
  incident to a given element.

  \begin{center}
  \begin{figure}[h]
    \T\begin{minipage}{6cm}
        \T\includegraphics[width=5cm]{bilder/cellcx-2triang.eps} 
        \W\htmlimage[ALT="a grid consisting of 2 adjacent triangles (2 cells, 5 edges, 4 vertices)"]{\img{cellcx-2triang}}
        \caption{A simple grid \ldots}
      \T\end{minipage}  \T\hspace{1cm}
    \T\begin{minipage}{6cm}
        \T\includegraphics[width=5cm]{bilder/cellcx-2triang-lattice.eps} 
        \W\htmlimage[ALT="the lattice of this grid"]{\img{cellcx-2triang-lattice}}
        \caption{\ldots and its lattice}
      \T\end{minipage}
    \end{figure}
  \end{center}

  For example, in the images, the grid consists of the following elements:
  vertices $v_1, \ldots, v_4$, edges $e_1, \ldots e_5$ and cells $c_1, c_2$.
  Their incidence relation is given by the lattice, where two different elements
  $x, y$
  are incident, if there is a path from  $x$ down to $y$  (the path is not allowed to
  visit the same level more than once). Therefore, elements of the same dimension
  are never incident, only adjacent. For example, $c_1$ is incident to $v_3$,
  and adjacent to $c_2$.

  These concepts give rise to a lot of different concrete types which are associated with a
  concrete grid type.
  Because this association is so fundamental and important,
  we use the special  {\em grid-types\/} entity to represent it.
  This is a class template which is specialized for concrete grid types,
  and thus acts as a mapping from  concrete grid types to the associated types introduced below.
  \xname{element-intro}
  \begin{Label}{element-intro}
  \introsubsection{Elements}
  \end{Label}

  Basically, a grid consists of a collection of {\em vertices\/}, {\em edges\/}
  etc. which are termed {\em k-Elements\/} (other common
  names are {\em k-Faces\/} or {\em k-cells\/}). 
  \\
  The notion is captured by the following types, named according to their dimension and their
  co-dimension.

      
  \begin{tabular}{lcc} \\ \hline
    \bf  k-Element &  \bf  Dimension &  \bf  Codimension  \\ \hline
    {\tt Vertex} &   0 &   -  \\
    {\tt Edge} &   1 &   -  \\ 
    {\tt Face} &   2 &   -  \\
    {\tt Facet} &   - &   1  \\
    {\tt Cell} &   - &   0  \\ \hline \\
  \end{tabular}
  
  Other common names are {\em node\/} (for vertex) and {\em volume\/} (for cell).
  These are not used here.
        
  The above naming scheme allows for a dimension-independent
  formulation of many algorithms: e.g. fluxes are always defined on {\tt Facets}
  (see \link{example}[~on page \pageref{flux-example}~]{flux-example}).
      
  In general, some of these types can coincide: for a 2D-grid, the types {\tt Edge}
  and {\tt Facet} are often the same. 
  Also,  the type {\tt Face} cannot be defined for 1D-grids.

  \begin{Label}{handle-intro}
    \introsubsection{Element handles}
  \end{Label}
   An element  implementation must be a class, 
   because there are some member functions
   required \sectionlink{(C)}{GridElement}. 
   A corresponding object must contain a reference to the underlying grid,
   which is redundant when considering sets of elements belonging to the same grid.
   For a minimal representation of elements, often just a number (a {\em element handle\/})
   is required, which allows unambiguous reconstruction of an element belonging to a given grid.
   
   Element handles can be used to represent sets of elements in an economic way,
   or to implement mappings between different grids (grid morphisms).

   \begin{Label}{sequence-it-intro}
     \introsubsection{Sequence iterators}
   \end{Label}
   At a basic level, a grid can be seen as a container of its elements, that is,
   as a sequence of vertices, edges and so on. Correspondingly, a grid defines
   sequence iterators \sectionlink{(C)}{GridSequenceIterator}
   to capture this property.

   These iterators satisfy the STL Forward Iterator requirements.

   \begin{figure}[h]
     \begin{center}
       \T\includegraphics{bilder/cell-it.eps}
       \W\htmlimage[ALT="A cell iterator hopping from cell to cell"]{\img{cell-it}}
       \caption{A cell iterator}
     \end{center}
   \end{figure}

   {\bf Examples:} 
   A loop over all vertices of a grid {\tt G} would read:
   \begin{example}
       my_grid_type g;
       // ...
       int nv = 0;
       for(\link{VertexIterator}{GridVertexIterator.html} v = g.FirstVertex(); ! v.IsDone(); ++v)
         nv++;
       ASSERT( nv == \link{g.NumOfVertices()}{VertexGridRange.html} );
\end{example}
      To draw the line-graph of a grid, we would do:

\begin{example}
       my_geometry geom(g);
       // ...
       for(\link{EdgeIterator}{GridEdgeRange.html} e = G.FirstEdge(); ! e.IsDone(); ++e)
         draw(geom.\link{segment}{GeomSegment.html}(*e));
\end{example}

\begin{Label}{incidence-it-intro}
    \introsubsection{Incidence iterators}
    \end{Label}
    Grids are more than just a collection of vertices, edges and so
    on. We want to be able to navigate through the neighbourhood of an k-element.
    For this purpose, we need {\em incidence iterators\/}. 
    For example we might want
    to access all vertices incident to a cell.

   \begin{figure}[h]
     \begin{center}
       \T\includegraphics{bilder/cell-on-cell.eps}
       \W\htmlimage[ALT="A cell-on-cell iterator loops through the 3 adjacent triangles of a triangle"]{%
        \img{cell-on-cell}}
      \caption{A cell-on-cell iterator}
     \end{center}
   \end{figure}

    In order to iterate 
    over the incident elements of a {\tt Cell}, 
    we can implement (and use) the following iterators:

    \begin{tabular}{ll} 
      \\
      {\tt VertexOnCellIterator} &  {\tt Cell::FirstVertex()}  \\ 
      {\tt EdgeOnCellIterator} &   {\tt Cell::FirstEdge()} \\ 
      {\tt FacetOnCellIterator} &  {\tt Cell::FirstFacet()}  \\
      {\tt CellOnCellIterator} &  {\tt Cell::FirstCell()} \\ 
      \\
    \end{tabular}

    The corresponding iterators for vertices have analogous names.
    Strictly speaking {\tt CellOnCellIterator}  is not an 
    incidence-iterator, because two
    neighboring cells are not incident.
    A typical example might be to calculate flows over cell boundaries.
    In pseudo-code, this reads:
    \begin{example}
     for all cells c of the grid g
        initialize the flux to 0
        for all facets fc  of the cell c
          add the flux over fc to the flux of c
    \end{example}
     
   Using our concepts, this translates into the following:
 
    \begin{Label}{flux-example}
 
  \begin{example}
 a_grid_type g;
 typedef \link{grid\_types}{grid_types.html}<a_grid_type> gt;
 // ....
 \sectionlink{grid\_function}{DataAssociationLayer}<\sectionlink{Cell}{GridCell},flux\_type> fluxes(g);
 for(gt::\sectionlink{CellIterator}{GridVertexIterator} c = g.FirstCell(); ! c.IsDone(); ++c) \{
   fluxes[*c]= flux\_type(0.0);
   for(gt::\sectionlink{FacetOnCellIterator}{VertexOnCellIterator} f = (*c).FirstFacet(); ! f.IsDone(); ++f)
     fluxes[*c] += calculate\_flow(f);
   \}
\end{example}
\end{Label}

Here {\tt flux\_type} depends on the application and typically is a vector
of low dimension.
