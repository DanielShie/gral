\xname{ArchetypedGrid}
\begin{Label}{ArchetypedGrid}
 \conceptsection{ArchetypedGrid Concept}
\end{Label}

\conceptsubsection{Description}
An \concept{ArchetypedGrid} is a grid which associates
an archetype with each cell. 
An archetype is a combinatorial model of a cell,
or, more precisely, of the boundary of a cell. 
Thus, the archetype $A$ of a cell $C$ of dimension $d$ is a grid of
dimension $d-1$. The cells of $A$ correspond to the facets of $C$.

The rationale for having archetypes stems from the fact that in general, 
it is not enough to know the vertices of a cell to determine its topology.
We must also know how the vertices form the facets of the cell.
In general, the archetype will again be a model of \concept{ArchetypedGrid},
so that we can apply the concept recursively.

For an example of how to use archetypes, consider the problem of copying grids
from one representation (\emph{source}) into a different one (\emph{target}). 
Consider the case of quadrilateral cells, 
where the vertices of a cell are ordered in a binary way in the source representation 
(for example, row-major numbering in a Cartesian grid), 
and cyclically in the target representation.
If we don't care about this different numbering, twisted cells will result.
In order to avoid this, we must somehow know about the different orderings.
We can of course implement ad hoc copy procedures for each pair of grid representations,
but this is against the ``generic spirit''. 

  \begin{figure}[h]
     \begin{center}
       \begin{Label}{vertex-ordering}
       \T\includegraphics{bilder/vertex-ordering}
       \W\htmlimage[ALT="Incompatible vertex-ordering may result in self-intersecting cells"]{%
          \img{vertex-ordering}}
        \caption{A binary ordering (left), an incompatible cyclic ordering (middle), 
                 and the effect of direct copying (right)}
        \end{Label}
     \end{center}
   \end{figure}

A better way is to make this ordering explicit and amenable to algorithmic treatment.
This is what archetypes are for.
Before starting the actual copying of the grid, 
we try to match the archetypes of the source grid to that of the target grid
by calculating isomorphisms between source and target archetypes.
These isomorphisms will define a permutation of the vertex ordering of the source cells,
such that an isomorphic copy of the source grid results.


\conceptsubsection{Refinement of}
\conceptlink{Grid}{Grid}
\\
\conceptlink{Cell-Vertex Input Grid Range}{Cell-VertexInputGridRange}


\conceptsubsection{Notation}
\type{G} is a type which is a model of \concept{ArchetypedGrid}\\
\variable{g} is an object of type \type{G}\\
\variable{A} is an object of type \type{G::archetype\_type} \\
\variable{a} is an object of type \type{G::archetype\_handle}\\
\variable{ai} is an object of type \type{G::archetype\_iterator}\\
\variable{C} is an object of type \type{G::Cell}
\variable{c} is an object of type \type{G::cell_handle}

\conceptsubsection{Definitions}

\conceptsubsection{Associated types}

\begin{tabularx}{14cm}{RRR} 
  \T \\   \hline
  {\bf  Name  } & {\bf  Expression  } & {\bf  Description  } \\ 
  \hline
  archetype type  & \code{G::archetype_type} &
  The type of cell archetypes, 
  \par model of \conceptlink{Grid}{Grid}
  and of \conceptlink{Cell-Vertex Input Grid Range}{Cell-VertexInputGridRange}
  of dimension \code{G::dimension()-1}
  \\
  \hline
  archetype handle & \code{G::archetype_handle} & unique handle for an archetype 
  within the grid \code{G}
  \\
  \hline
  archetype iterator & \code{G::archetype_iterator} & iterator over all archetypes of a grid 
  \T \\   
\end{tabularx}

\conceptsubsection{Valid Expressions}

\noindent
\begin{tabular}{llll} 
  \T \\  \hline
  \bf  Name  &\bf  Expression  &\bf  Type requirements  & \bf  return type  \\ 
  \T \hline \\
  Archetype association &
  \code{g.Archetypeof(C)} &
   ~ & 
  \type{archetype\_type const\&} \\
  \T \hline \\
  Archetype association &
  \code{g.archetype\_of(c)} &
   ~ & 
  \type{archetype\_handle} \\
  \T \hline  \\
  Archetype iteration start&
  \code{g.BeginArchetype()} &
   ~ & 
  \type{archetype\_iterator} \\
  \T \hline \\
  Archetype iteration past-the-end &
  \code{g.EndArchetype()} &
   ~ & 
  \type{archetype\_iterator} \\
  \T \hline \\
  Archetype count &
  \code{g.NumOfArchetypes()} &
   ~ & 
  \type{unsigned} \\
  \T \hline \\
  Handle to archetype &
  \code{g.Archetype(a)} &
   ~ & 
  \type{archetype\_type const\&} \\
  \T \hline  \\
  iterator to handle &
  \code{g.handle(ai)} &
   ~ & 
  \type{archetype\_handle} 
  \T \\  \hline  \\
\end{tabular}

\conceptsubsection{Expression semantics}

\noindent
\begin{tabularx}{14cm}{RRRRR} 
  \T \\ \hline
  \bf  Name       &
  \bf  Expression &
  \bf  Precondition&
  \bf   Semantics &
  \bf   Postcondition \\
  \hline
  \T \\  \hline  \\
\end{tabularx}

\conceptsubsection{Invariants}

\noindent
\begin{tabularx}{14cm}{RR}
  \T \\  \hline
   & \code{\&(g.Archetype(archetype\_of(C))) == \&(g.ArchetypeOf(C))} \\ 
   & \code{std::distance(g.BeginArchetype(), g.EndArchetype()) == g.NumOfArchetypes()} \\
   & \code{ \&(*ai) == \&(g.Archetype(g.handle(ai)))} 
  \T \\ \hline
\end{tabularx}

\conceptsubsection{Refinements}

\conceptsubsection{Models}
\sectionlink{\type{Complex2D}}{Complex2D}   

\conceptsubsection{Notes}
\begin{enumerate}
%\item
\end{enumerate}

\conceptsubsection{See also}


