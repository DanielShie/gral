\xname{GridFunction}
\begin{Label}{GridFunction}
\conceptsection{Grid Function Concept}
\end{Label}

\conceptsubsection{Description}

The  {\em Grid Function\/} concept refines the 
\sectionlink{\concept{ Grid Element Function}}{GridElementFunction} concept
in that it bind the function
to a particular grid. 
Evaluating  a grid function without a valid grid set or
with an element whose grid is different from that of the grid function
is considered an error.

By binding to a particular grid, it is possible to treat both domain
and range of a grid function as  sequences with associated iterators.

\conceptsubsection{Refinement of}
\sectionlink{\concept{Grid Element Function}}{GridElementFunction}

\conceptsubsection{Notation}
{\tt F} is a type which is a model of  Grid  Function 
\\
{\tt f} is an object of type  {\tt F}
\\
{\tt i} is an object of  {\tt F::const\_iterator}
\\
{\tt e} is an object of  {\tt F::element\_type}
\\
{\tt n} is an object of  {\tt F::size\_type}
\\
{\tt G} is shorthand for  {\tt F::grid\_type}

\conceptsubsection{Definitions}
\label{bound}
A grid function {\tt f} is {\em bound\/} to a grid {\tt g},
if {\tt \&g == \&f.TheGrid()}. Else  {\tt f} is {\em unbound\/}.

\label{range}    
The {\em range\/} of a grid function {\tt f} is the set of all elements of type
{\tt F::element\_type} in {\tt f.TheGrid()}.

\label{domain}
The {\em domain\/} of a grid function is the set of all values of the form {\tt f(e)} 
where {\tt e} is in the range of {\tt f}.

\conceptsubsection{Associated types}

\begin{tabularx}{14cm}{llR}
  \hline
  \bf  Name  &\bf  Expression  &\bf  Description   \\ 
  \hline
  Grid type & 
  {\tt F::grid\_type} &
  type of the corresponding associated grid,
  \par model of \sectionlink{\concept{ Grid}}{Grid} 
  \\ 
  value iterator &
  {\tt F::const\_iterator} &
  iterator over the values of {\tt f}
  \par model of STL
  \stllink{Forward Iterator}{ForwardIterator}
  \\ 
  size type & 
  {\tt F::size\_type} &
  integral type capable of representing the possible sizes
  of the value sequence.
  \\ 
  \hline
  \\
\end{tabularx}

\conceptsubsection{Valid Expressions}
\begin{tabularx}{14cm}{RRll} 
  \hline
  \bf  Name  &\bf  Expression  &\bf  Type requirements  & \bf  return type  \\
  \hline
  Grid reference  &
  {\tt f.TheGrid();} &
  ~ &
  {\tt F::grid\_type const\&} 
  \\ 
  start of value sequence  &
  {\tt f.begin();} &
  ~ &
  {\tt F::const\_iterator} 
  \\ 
  end of value sequence  &
  {\tt f.end();} &
  ~ &
  {\tt F::const\_iterator} 
  \\ 
  size of value sequence  &
  {\tt f.size();} &
  ~ &
  {\tt F::size\_type} 
  \\ 
  \hline
  \\
\end{tabularx}

\T\begin{small}
\conceptsubsection{Expression semantics}
\W\begin{tabularx}{15cm}{RRRRR} 
\T\begin{tabularx}{15cm}{>{\raggedright\arraybackslash}p{1.5cm}RRRR}
  \hline
  \bf  Name     &
  \bf  Expression &
  \bf  Precondition&
  \bf  Semantics &
  \bf  Postcondition
  \\ 
  \hline
  Grid reference  &
  {\tt G\& g = f.TheGrid();} &
  {\tt f} is bound to a grid  &
  get reference to the underlying grid & 
  {\tt \&(g.TheGrid()) == \&(f.TheGrid())} 
  \\ 
  start of value sequence  &
  {\tt i = f.begin();} &
  {\tt f} is bound to a grid  &
  return iterator to start of value sequence &
  {\tt \xlink{distance}{\STLURL/distance.html} (i,f.end()) == f.size()}
  \\ 
  end of value sequence  &
  {\tt i = f.end();} &
  {\tt f} is bound to a grid  &
  return iterator to past-the-end of value sequence &
  {\tt i} is past-the-end, {\tt i == f.end()}
  \\ 
  size of value sequence  &
  {\tt n = f.size();} &
  {\tt f} is bound to a grid  &
  return size of value sequence &
  {\tt f.size() == 
    \xlink{distance}{\STLURL/distance.html} (f.begin(),f.end())}
  \\ 
  \hline
  \\
\end{tabularx}
\T\end{small}

\conceptsubsection{Refinements}
\sectionlink{\concept{ Mutable Grid Function }}{MutableGridFunction}

\conceptsubsection{Models}
\sectionlink{{\tt cell2handle\_map<G>}}{cell2handle-map} 
defined in
\gralfilelink{grid-functors}{h}{base}
(\noteref{note-functors})
\\
\sectionlink{{\tt vertex2coord\_map<Geom>}}{vertex2coord-map}
defined in
\gralfilelink{geometry-functors}{h}{base}
(\noteref{note-functors})

\conceptsubsection{Notes}
\begin{enumerate}
\item \notelabel{note-functors}
       Currently, iteration and size is not supported in these examples.
\end{enumerate}

\conceptsubsection{See also}
\sectionlink{\concept{ Grid Element Function }}{GridElementFunction} ~
  

