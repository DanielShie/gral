\xname{GridElementHandle} 
\begin{Label}{GridElementHandle} 
 \conceptsection{Grid Element Handle Concept}
\end{Label}

\conceptsubsection{Description}
A {\em Grid Element Handle\/} is a minimal representation of  a
\sectionlink{\concept{Grid Elements}}{GridElement}. 
They are unique only within a single 
\sectionlink{\concept{Grid}}{Grid}, 
which allows to map back and forth between handles and their
corresponding elements.
 

\conceptsubsection{Refinement of}
STL \xlink{Assignable}{http://www.sgi.com/Technology/STL/Assignable.html}
\\
STL \xlink{Equality Comparable}{http://www.sgi.com/Technology/STL/EqualityComparable.html}



\W\conceptsubsection{Notation}

\W\conceptsubsection{Valid Expressions}

\W\conceptsubsection{Expression semantics}

\conceptsubsection{Refinements}
\conceptlink{Grid Vertex Handle}{GridVertexHandle}
\\
\conceptlink{Grid Edge   Handle}{GridVertexHandle}
\\
\conceptlink{Grid Facet  Handle}{GridVertexHandle}
\\
\conceptlink{Grid Cell   Handle}{GridVertexHandle}
\\

\W\conceptsubsection{Models}

\conceptsubsection{Note}
\begin{enumerate}
\item \notelabel{note-handles-are-builtin}
  Typical models of \concept{Element Handles} are basic built-in types, 
  such as integral types or pointers.
  As such, it is not required that the types of handles corresponding to different element types
  (such as vertices and cells) be themselves different.
\item
  Often, handles have to fullfil additional requirements. 
  For example, implementations of \conceptlink{Container Grid Functions}{ContainerGridFunction}
  often exploit special properties of handles, such as being consecutively ordered integral types,
  or being a hashable type (that is, the \stltypelink{hash}{hash} template has been specialized
  for them).
\end{enumerate}

\conceptsubsection{See also}
\sectionlink{\concept{Grid}}{Grid} ~
\sectionlink{\concept{ Grid Entity}}{GridEntity} ~
\sectionlink{\concept{ Grid Element}}{GridElement} ~


