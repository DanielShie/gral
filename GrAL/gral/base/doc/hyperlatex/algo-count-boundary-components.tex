\xname{count-boundary-components}
\begin{Label}{count-boundary-components}
\algosection{Count boundary components}
\end{Label}

\algosubsection{Declaration}

\begin{example}
template<class GRID>
int count\_boundary\_components(GRID const& G);
\end{example}

\algosubsection{Description}
Counts the number of connected components of the boundary of {\tt G}.
\algosubsection{Definition}
Declared in \nmwrcodelink{find-boundary-components.h}{Grids/Algorithms}\\
Defined  in \nmwrcodelink{find-boundary-components.C}{Grids/Algorithms/generic}\\

\algosubsection{Type requirements}
\type{GRID} is a model of 
\Conceptlink{Grid} and of
\conceptlink{Facet Grid Range}{VertexGridRange}.\\
There is a \conceptlink{Partial Grid Function}{PartialGridFunction}
container on facets
associated to \type{GRID} \noteref{note-cbc-partial-gf}.
\algosubsection{Preconditions}
\Var{G} represents a \Glossarylink{manifold-with-boundary} grid.
\algosubsection{Postconditions}
The value returned is the number of connected components of \Var{G}'s boundary.
\algosubsection{Complexity}
Expected time $O(F)$, where $F$ is the number of facets of \Var{G}.
\algosubsection{Example}

\algosubsection{Uses}
 \sectionlink{find-boundary-component-germs}{find-boundary-component-germs}.

\algosubsection{Used in}

\algosubsection{Notes}
\begin{enumerate}
\item \label{note-cbc-partial-gf}
This is not a real constraint, because there is a 
\sectionlink{generic implementation}{partial-grid-function-hash}
of partial grid functions.
The only thing needed to use it is to define (specialize)
the \xlink{\type{hash<>}}{\STLURL/hash.html}
template on \type{Grid::Facet}.
\end{enumerate}
\algosubsection{See also}

