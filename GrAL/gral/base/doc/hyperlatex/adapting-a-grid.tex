\xname{AdaptingToInterface}
\begin{Label}{AdaptingToInterface}
\introsection{Creating an Interface Adaption to Your Own Grid Data Structure}
\end{Label}


If you want to use GrAL, you typically want to use it's generic
components and less it's concrete grid data structure
-- at least that's how it is intended.
Therefore, you will need to create an interface layer 
satisfiying the syntax specified in the \Sectionlink{Concepts} section.

For sake of simplicity, let us assume that your grid data structure
is a plain  good ole C one:
Cell-vertex incidences are stored in a plain array \texttt{cells},
such that \texttt{cells[3*c +v]} gives the index of the \texttt{v}'th vertex
of cell \texttt{c}. 
Likewise, an array \texttt{geom} 
holds the coordinates of vertex \texttt{v}
at indices \texttt{2*v} and \texttt{2*v+1}.

\begin{Label}{triang2d-combi}
\introsubsection{Defining the Combinatorial Layer}
\end{Label}

Let's call our grid class \texttt{triang2d}:
\begin{example}
class triang2d \{
  int * cells;

  // to be continued ...
\};
\end{example}
From the information present in the array \texttt{cells},
we can define  the following types more or less directly:
\texttt{Vertex, VertexIterator, Cell, CellIterator, VertexOnCellIterator}.
We will implement them as nested classes of \texttt{triang2d}.
Also, in order to save some work, we will implement the
\Conceptlink{Vertex} and \Conceptlink{VertexIterator} by the same class
\texttt{triang2d::Vertex},
and the same goes for \Conceptlink{Cell}.
Finally, element handles are just integers in this case.

\begin{example}
class triang2d \{

  typedef int vertex_handle;
  typedef int cell_handle;

  class Vertex \{
     // to be continued ...
  \};
  typedef Vertex VertexIterator;

  class Cell \{
     // to be continued ...
  \};
  typedef Cell   CellIterator;

  class VertexOnCellIterator \{
     // to be continued ...
  \};
\};
\end{example}
We first show the definition of \texttt{triang2d::Cell},
and then \texttt{triang2d::VertexOnCellIterator}.

\begin{example}
class triang2d \{
   class Cell \{
     typedef Cell     self;
     typedef triang2d grid_type;
   private:
      cell_handle      c;
      grid_type const* g;
   public:
      Cell(grid_type const& gg, cell_handle cc = 0) : g(&gg), c(cc) \{\}  
   
      self      & operator++() \{ ++c; return *this;\}
      self const& operator* () const \{ return *this;\}
      bool IsDone() const \{ return c >= TheGrid.NumOfCells();\}
  
      cell_handle handle() const \{ return c;\}     
      grid_type   const& TheGrid() const \{ return *g; \} 

      unsigned NumOfVertices() const \{ return 3; \}
      VertexOnCellIterator FirstVertex() const 
       \{ return VertexOnCellIterator(*this);\}
   \};

\};
\end{example}
(In pactice, we would have to define \texttt{FirstVertex()} outside
of \texttt{triang2d::Cell}.)
The class \texttt{triang2d::Vertex} is of course very similar and 
not shown here. 
Now for the incidence iterator which models \Conceptlink{VertexOnCellIterator}:

\begin{example}
class triang2d \{
 
  class VertexOnCellIterator \{
    // typedefs omitted ...
  private:
    cell_handle c;
    int         lv;
    grid_type   g;
  public:
    VertexOnCellIterator(Cell const& cc) : c(CC.c), lv(0), g(CC.g) \{\}
     
    self&  operator++() \{ ++lv; return *this; \}
    Vertex operator*() const \{ return Vertex(*g, handle());\}
    bool   IsDone() const \{ return (lv >= 3);\}
    
    vertex_handle handle()  const \{ return g->cells[3*c+lv];\}
    Cell          TheCell() const \{ return Cell(*g,c);\}
    grid_type     TheGrid() const \{ return *g;\}
  \};

\}
\end{example}
Now we have defined enough types to satisfy the 
\Conceptlink{Cell-VertexInputGridRange} model.
It remains to fill in the \texttt{grid\_types<>} template:
\begin{example}
template<>
struct grid_types<triang2d> \{
  typedef triang2d grid_type;

  typedef triang2d::vertex_handle vertex_handle;
  
  //  etc.
\};
\end{example}


\begin{Label}{triang2d-geom}
\introsubsection{Defining the Geometrical Layer}
\end{Label}

To be written.

\begin{Label}{triang2d-gf}
\introsubsection{Defining the Grid Function Layer}
\end{Label}

To be written.

\begin{Label}{triang2d-gf}
\introsubsection{Defining the Mutating primitives}
\end{Label}

To be written.
