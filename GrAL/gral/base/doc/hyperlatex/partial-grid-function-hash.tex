\xname{partial-grid-function-hash}
\begin{Label}{partial-grid-function-hash}
\datasection{partial\_grid\_function\_<E,T> }
\end{Label}

\datasubsection{Declaration}
\begin{example}
template<class E, class T>
class partial_grid_function;
\end{example}
\datasubsection{Description}
The class template \type{partial\_grid\_function} is an implementation 
of the \conceptlink{Partial Grid Function}{PartialGridFunction} concept.
It works for any element type, provided there 
is a specialization of the \stltypelink{hash}{hash} template for 
this type defined.
\datasubsection{Model of}
\conceptlink{Partial Grid Function}{PartialGridFunction}
\datasubsection{Definition}
Defined in \gralcodelink{partial-grid-function-hash.h}{base}

\datasubsection{Template parameters}
\begin{tabular}{lll} \hline
  \bf Parameter & \bf Description & \bf Default \\
  \hline
  \type{E}  & the element type  & ~ \\
  \type{T}  & the value  type  & ~ \\
  \hline
\end{tabular}

\datasubsection{Type requirements}
\type{E} must be a model of \conceptlink{Grid Element}{GridElement}.\\
The \stltypelink{hash}{hash} template must be specialized for \type{E}. \\
\type{T} must be a model of STL \Stllink{Assignable}.

\datasubsection{Public base classes}
\type{grid\_function\_hash\_base<E,T>}, defined in
\gralcodelink{grid-function-hash.h}{base}.

\datasubsection{Members}
\datasubsection{New members}

\datasubsection{Example}
\begin{example}
 a_grid_type g;
 
 typedef grid_types<a\_grid\_type> gt;
 grid_function_vector<gt::Edge, double> length(g);
 for(gt::EdgeIterator e(g); ! e.IsDone(); ++e)
   length[*e] = distance(geom.coord((*e).V1()),
                         geom.coord((*e).V2()));
\end{example}
\datasubsection{Known uses}
Used for implementation of total grid functions for 
\type{Complex2D::Edge}.

\datasubsection{Notes}

\datasubsection{See also}
\Sectionlink{grid-function-vector}
~
\Sectionlink{grid-function-hash}
