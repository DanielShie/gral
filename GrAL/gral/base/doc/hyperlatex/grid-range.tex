\xname{GridRange}
\begin{Label}{GridRange}  
  \conceptsection{Grid Range Concept}
\end{Label}

    \conceptsubsection{Description}
    A {\em  Grid Range} is a part of a grid, its {\em  base grid}.
    The underlying mathematical concept
    is that of (a subset of) a (finite) 
    \xlink{CW-complex}{http://www.treasure-troves.com/math/CW-Complex.html}.
    Some well-known specializations of this concept are
    \xlink{triangulations}{http://www.treasure-troves.com/math/Triangulation.html}
    boundary complexes of convex 
    \xlink{polytopes}{http://www.treasure-troves.com/math/Polytope.html} 
    and regular Cartesian grids.

    A Grid Range behaves in most circumstances like a 
%    \xlink{Grid}[~(section \ref{Grid})~]{Grid.html}. 
    \sectionlink{\concept{Grid}}{Grid}
    The main difference is that a Grid Range has reference semantics with respect to
    its underlyin base grid, that is, the incidence relationship is determined by the 
    base grid.
    This influences the behaviour of
    \sectionlink{\concept{Incidence Iterators}}{GridIncidenceIterator}
    associated with a Grid Range,
    which may  visit 
    \sectionlink{\concept{grid elements}}{GridElement}
    that are contained in the base grid,
    but not  in the grid range    
    \noteref{note-incidence}
%    \link*{\arabic{\value{notecounter}}}[\Ref]{note-incidence}).
    
    A Grid is a special case of a Grid 
    Range \noteref{note-reference} %\link*{\ref{note-reference}}[~(see note \Ref)~]{note-reference}.

    {\em  NOTE:} A grid range as such does offer almost no functionality at all.
    Any useful model will be a specialization of one or more 
    element ranges 
    \noteref{note-element-types}
    %\link*{\ref{note-element-types}}[~(see note \Ref)~]{note-element-types}
    like     
    \sectionlink{\concept{Grid Vertex Range}}{VertexGridRange}.

    \conceptsubsection{Refinement of}
     STL \xlink{Assignable}[]{http://www.sgi.com/Technology/STL/Assignable.html}


    \conceptsubsection{Notation }
    {\tt  R} is a type which is a model of grid range
    \\
    {\tt  r} is an object of type {\tt  R}
    \\
    {\tt G} is {\tt R::grid\_type}

    \conceptsubsection{Associated types}\noteref{note-gridtypes}

    \begin{tabular}{lll} \\
      \hline
      \bf  Name  &\bf  Expression  &\bf  Description   \\ 
      \hline
      base grid &
      {\tt  R::grid\_type}  &
      type of the  ranges' base grid 
      \\ 
      \hline
      \\
    \end{tabular}
 

    \conceptsubsection{Valid Expressions }
   \begin{tabular}{llll} \\
     \hline
       \bf  Name  &\bf  Expression  &\bf  Type requirements  & \bf  return type  \\ 
       \hline
       base grid reference  &
       {\tt  r.TheGrid()}  &
       & 
       {\tt  grid\_type const\&}  \\ 
       \hline
       \\
    \end{tabular}

   
    \conceptsubsection{Expression semantics }
    \begin{tabularx}{14cm}{RRRRR} \\
      \hline
      \bf  Name       &
      \bf  Expression &
      \bf  Precondition&
      \bf   Semantics &
      \bf   Postcondition
      \\
      \hline
      base grid   reference  &
      {\tt G const\&  g = r.TheGrid()}  &
      {\tt r } has not been default constructed &
      get the grid {\tt  r} references  &
      {\tt  \&g == \&(g.TheGrid())} {\tt == \&(r.TheGrid())}  \\ 
    \hline
    \\
  \end{tabularx}

  \W\conceptsubsection{Complexity guarantees }

    \conceptsubsection{Refinements}
    \sectionlink{\concept{Vertex Grid Range}}{VertexGridRange} ~
    \W\\
    \sectionlink{\concept{Edge Grid Range}}{VertexGridRange} ~
    \W\\
    \sectionlink{\concept{Facet Grid Range}}{VertexGridRange} ~
    \W\\
    \sectionlink{\concept{Cell Grid Range}}{VertexGridRange} ~
    \W\\
    
    \conceptsubsection{Models }
    \link{{\tt enumerated\_grid\_range} }[~(\Ref)~]{EnumeratedGridRange} ~
    \W\\
    \link{{\tt Complex2D}}[~(\Ref)~]{Complex2D} ~
    \W\\
    \link{{\tt  Complex2DInput}}[~(\Ref)~]{Complex2DInput} ~

    \conceptsubsection{Notes}
    \begin{enumerate}
    \item       \notelabel{note-reference}
      If {\tt  R} is a model of
      \sectionlinkshort{\concept{Grid}}{Grid},
      then {\tt  R::grid\_type} is identical to {\tt  R}.
      An object {\tt  r} of type {\tt R} 
      then references itself via {\tt  r.TheGrid()}, 
      that is, it has value semantics.
          
      \item   \notelabel{note-gridtypes}
        Technically, these types are bundled in a struct {\tt  grid\_types<R>}
      which is used by the algorithms to access these types. This opens up the possibility
      to parameterize algorithms by such a {\em  traits class} like {\tt  grid\_types<R>},
      thereby introducing different iterator and element types, for example counting iterators
      or debug iterators producing graphical output.
    
      In this case, it would be more precise to say that one 
      ``associates types with {\tt  R}'', 
      instead of speaking of ``types associated with {\tt  R}''.

      
      \item \notelabel{note-incidence}
        This may seem as an odd behaviour. However, this faithfully reflects what many
      locally operating grid-based algorithms are intuitively expected to do when
      given a proper subrange of a grid: The range restricts the region where some
      work is to be done, but on each element, the algorithm accesses also some
      meighboring elements via incidence iterators, which may or may not belong to the
      range. The most striking example for this occurs if grids are distributed with
      some overlap: On each part, the algorithm works only on the locally owned range, 
      but it accesses also elements in the overlap which are copied from other parts.

     
      \item 
        \notelabel{note-element-types}
        It is not necessary to require all possible element types to be defined.
      Useful examples are Input Grids, 
      such as \link{ {\tt  Complex2DInput} }[~(\Ref )~]{Complex2DInput}
      which are used just to read a grid from
      a specific file format.
    \end{enumerate}

    \conceptsubsection{See also }
    \link{Grid }[~(\Ref )~]{Grid} ~
    \link{Grid Element }[~(\Ref )~]{GridElement} ~
    \link{Sequence Iterator }[~(\Ref )~]{SequenceIterator} ~
    \link{Incidence Iterator }[~(\Ref )~]{IncidenceIterator} 


