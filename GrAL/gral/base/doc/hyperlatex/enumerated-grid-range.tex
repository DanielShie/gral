\xname{EnumeratedGridRange}
\begin{Label}{EnumeratedGridRange}
\datasection{enumerated\_subrange}
\end{Label}

\datasubsection{Declaration}
\begin{example}
template<class G>
class enumerated\_subrange;
\end{example}

\datasubsection{Description}
The  class template \type{enumerated\_subrange} implements
a sub-range of a grid.
Cells and vertices must be explicitly joined 
to the range\noteref{note-vertex-cell-consistency}.
Iteration over facets is provided.

\datasubsection{Model of}
\sectionlink{\concept{Vertex Grid Range}}{VertexRange}
\\
\sectionlink{\concept{Edge Grid Range}}{EdgeRange}\noteref{note-edgerange-in-2d-only}
\\
\sectionlink{\concept{Facet Grid Range}}{FacetRange}
\\
\sectionlink{\concept{Cell Grid Range}}{CellRange}

\datasubsection{Definition}
Defined in \gralfilelink{enumerated-subrange}{h}{base}
\datasubsection{Template parameters}

\begin{tabular}{lll} \hline
  \bf Parameter & \bf Description & \bf Default \\
  \hline
  {\tt G}  & the base grid  & ~ \\
  \hline
\end{tabular}

\datasubsection{Type requirements}
{\tt G} must be a model of \sectionlink{\concept{Grid}}{Grid}.

\datasubsection{Public base classes}
None.
\datasubsection{Members}

\datasubsection{New members}

\begin{tabularx}{11cm}{lR}
  \hline
  \bf Member & \bf Description \\
  \hline
   \multicolumn{2}{c}{\bf\em Types} \\
   \hline
   {\tt vertex\_range\_ref } 
   & reference to vertex range, 
   \par defined as {\gralclasslink{vertex\_range\_ref}{base}<G,R>} \\
   {\tt cell\_range\_ref } 
   & reference to cell range, 
   \par defined as {\tt \gralclasslink{cell\_range\_ref}{base}<G,R>} \\
   \hline
   \multicolumn{2}{c}{\bf\em Functions} \\
   \hline
  {\tt append\_vertex(vertex\_handle v)} & add a new vertex to {\tt r} \\
  {\tt append\_cell(cell\_handle c)} & add a new cell to {\tt r} \\
  {\tt vertex\_range\_ref vertices()}  & reference to vertex range \\
  {\tt cell\_range\_ref cells()}  & reference to cell range \\
  \hline
\end{tabularx}

\datasubsection{Example}
\begin{example}
  a\_grid\_type g;
  ...
  enumerated\_subrange<a\_grid\_type>  r(g); // empty range
  // fill with cells
  for(gt::CellIterator c(g); ! c.IsDone(); ++c)
    if(predicate(*c)) // some predicate on cells
      r.append_cell(c.handle());
  // determine vertex set of r.cells()
  \gralfunctionlink{ConstructSubrangeFromCells}{base}(r,r.cells());
  
\end{example}
\datasubsection{Notes}
\begin{enumerate}
\item \notelabel{note-vertex-cell-consistency}
  The availability of these operations means that
  instances of  \type{enumerated\_subrange}
  do not ensure that their 
  vertex range is the vertex set of the 
  their cell range.
  It is the responsibility of the client to ensure this 
  {\em if\/} it is needed.
  If this equality is not satisfied, it simply means that the range
  does not represent a \Glossarylink{dimension-homogeneous} grid.

  Note, however, that there are components that can help in ensuring
  this property in a subrange, for example the 
  \gralfunctionlink{ConstructSubrangeFromCells}{base}  algorithm.


\item \notelabel{note-edgerange-in-2d-only}
  The  \sectionlinkshort{\concept{Edge Grid Range}}{EdgeRange} concept
  is supported in 2D only (when edges and facets coincide).
  The current implementation simply assumes that the template parameter
  {\tt G} is two-dimensional; for future extension, 
  a compile-time switch on the dimension should be performed here.
\item \notelabel{note-element-types-of-subrange}
  The element types and incidence iterator types of
   \type{enumerated\_subrange} are forwarded from the template
   parameter \type{G}. There is currently no mechanism implemented
   to determine at compile time which types are provided by \type{G},
   therefore, some default assumption has to be made, excluding 
   some simpler grid types from the set of possible parameters.
   This restriction is artificial and should be removed.
\end{enumerate}

\datasubsection{See also}

\gralclasslink{enumerated\_element\_range}{base} ~
\gralclasslink{enumerated\_subrange}{base} ~
\gralclasslink{enumerated\_subrange\_ref}{base} ~
\sectionlinkshort{\concept{Grid Range}}{GridRange}