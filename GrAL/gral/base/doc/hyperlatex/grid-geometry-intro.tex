
\xname{GeometricalLayer}
\begin{Label}{GeometricalLayer}
\introsection{The geometric layer}
\end{Label}

\needswork{This documentation is still incomplete!}

Until now, we have seen grids as purely combinatorial constructs. 
Adding geometric functionality in a {\em separate\/} layer
has the advantage to make a combinatorial grid usable in a 
broader context, 
because there may be many different geometric embeddings for
a grid. 
Also, there are a lot of algorithms that do not require any geometry at all.

Some of the major points of variation of the \emph{mathematical\/} aspects
of geometric embeddings are
\begin{itemize}
\item  embedding in higher-dimensional space (2D grid in 3D space)
\item embedding in curved geometry (e.g. on a sphere)
\item non-linear elements
\item time-dependent embeddings 
\end{itemize}

Furthermore, geometric embeddings can differ in \emph{computational\/} aspects:
\begin{itemize}
\item arithmetic: exact or floating-point?
\item storage or calculation of entities?
\item exact or approximate calculation (e.\ g.\ cell volume
  for nonlinear embeddings)
\end{itemize}

   \begin{figure}[h]
     \begin{center}
       \T\includegraphics{bilder/zwei-geometrien.eps}
       \W\htmlimage[ALT="A linear and a non-linear geometry for the same grid"]{\img{zwei-geometrien}}
      \caption{A linear and a non-linear geometry for the same grid}
     \end{center}
   \end{figure}


What kind of functionality is available in a class implementing a 
geometric embedding for a grid?
This question cannot be answered in general.
The most basic geometry concept is that of \conceptlink{Vertex Grid Geometry}{VertexGridGeometry},
which just allows access to vertex coordinates.
A more advanced concept is 
\conceptlink{Volume Grid Geometry}{VolumeGridGeometry},
which defines geometric counterparts for all combinatorial grid elements,
as shown in the table.

\begin{tabular}{ll}\\
  \hline
  \multicolumn{2}{c}{\bf types}  \\ 
  \hline
  {\tt typedef coord\_type} \conceptlink{(C)}{GeomCoord} & point in space  \\
  {\tt typedef segment\_type} \conceptlink{(C)}{GeomSegment} &  1D segment corr. to {\tt Edge} \\ 
  {\tt typedef polygon\_type} \conceptlink{(C)}{GeomPolygon}    &  2D polygon corr. to {\tt Face} \\ 
  {\tt typedef polyhedron\_type} \conceptlink{(C)}{GeomPolyhedron} &  3D polyhedron corr. to {\tt Cell} (in 3D Grids)\\
  \hline
  \multicolumn{2}{c}{\bf functionals from combinatorics to geometry} \\ 
  \hline
  \begin{tabular}{l}
    {\tt coord\_type  coord(Vertex)} \\
    {\tt segment\_type segment(Edge)} \\
    {\tt polygon\_type polygon(Face)} \\
    {\tt polyhedron\_type polyhedron(Cell)} 
  \end{tabular}
  &
  \begin{tabular}{p{5cm}} 
    mappings of combinatorial
    entities to geometric entities 
  \end{tabular} \\
  \hline
  \\
 \end{tabular}

There may be a lot more functionality available, depending on the
actual geometry. For example, if the combinatorial and the geometric
dimension are equal, one may define outward normal in
the centers of facets.

