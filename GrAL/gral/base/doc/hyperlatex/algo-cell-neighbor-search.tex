\xname{cell-neighbor-search}
\begin{Label}{cell-neighbor-search}
\algosection{Cell-neighbor-search}
\end{Label}

\algosubsection{Declaration}
\begin{example}
[1] template<class NBF, class CELLRANGE>
    void CalculateNeighborCells
          (NBF             &  Nb,   // out
           CELLRANGE  const&  C);   // in

[2] template<class NBF, class CELLRANGE, class FACETMAP>
    void CalculateNeighborCells
          (NBF             &  Nb,   // out
           CELLRANGE  const&  C,    // in
           FACETMAP        &  F);   // inout

[3] template<class NBF, class CELLRANGE, class FACETMAP, class CGT>
    void CalculateNeighborCells
         (NBF             &  Nb,    // out
          CELLRANGE  const&  C,     // in
          FACETMAP        &  F,     // inout
          CGT        const&);       // in (only type used) 
\end{example}


\algosubsection{Description}
\function{CalculateNeighborRange} takes a range of cells \variable{C} 
and determines
the neighbor relation on its cells.
(Two cells are neighbors if they share a facet.)
The output is stored in \variable{nb}.

\function{CalculateNeighborRange} (versions [2], [3]) 
can be used incrementally
in the variable \variable{C}:
If $C = C_1 \cup \ldots \cup C_n$ is a partition of a given cell set,
then calling \function{CalculateNeighborRange} with the cell set $C$ is
equivalent to the successive calls with cell sets $C_1, \ldots, C_n$.
(The remaining arguments stay identical.)

\algosubsection{Definition}
\function{CalculateNeighborRange} is defined in 
\gralfilelink{cell-neighbor-search}{h}{base}

\algosubsection{Requirements on types}
\begin{itemize}
\item \templateparam{CELLRANGE}, \templateparam{CGT}:\\
  \templateparam{CELLRANGE} is a model of 
  \sectionlink{\concept{CellGridRange}}{VertexGridRange}. 

  The following types must be 
  \glossarylink{published}{publishing a type}
  by 
  the type \templateparam{CGT} (version [3])
  or 
  the template \type{grid\_types<CELLSET>} (versions [1], [2]):
 
  \begin{itemize}
  \item \type{Cell}: a model of 
    \sectionlink{\concept{Grid Cell}}{GridCell} 
    and
    \sectionlink{\concept{Facet Grid Range}}{VertexGridRange}
  \item \type{FacetOnCellIterator}: a model of 
    \sectionlink{\concept{Facet-On-Cell Iterator}}{Vertex-On-CellIterator}.
  \item \type{Facet}: a model of
    \sectionlink{\concept{Grid Facet}}{GridFacet} and of 
    \sectionlink{\concept{Vertex Grid Range}}{VertexGridRange}
  \item \type{Vertex}: a model of \sectionlink{\concept{Grid Vertex}}{GridVertex}.
  \end{itemize}

  In the following, we use
  \templateparam{CGT} also for \footlink{\type{grid\_types}}{intro-grid-types}\type{<CELLSET>} 
  in versions [1], [2].
\item \templateparam{NBF} \\ 
  is a model of 
  \sectionlinkUNDEF{\concept{Mutable Mapping}}{MutableMapping}
  from \type{CGT::FacetOnCellIterator} to \type{CGT::cell\_handle}.
\item \templateparam{FACETMAP} \\
  is a model of 
  \sectionlinkUNDEF{\concept{Mutable Mapping}}{MutableMapping}
  from 
  \sectionlink{\type{vtuple}}{vtuple}\type{<CGT::grid\_type>}
  to \type{CGT::FacetOnCellIterator}.
  In addition, the following expressions must be defined:
  (Here {\tt i} is of type \type{FACETMAP::iterator}, {\tt f}
  is of type \type{CGT::FacetOnCellIterator}, and
  {\tt F} is of type \type{FACETMAP}. )
  \begin{itemize}
  \item type \type{iterator}
  \item {\tt i = F.end();}
  \item {\tt i = F.find(f);}
  \item {\tt F.erase(i);}
  \end{itemize}
  This is for example fullfilled if \templateparam{FACETMAP}
  is a model of STL \stllink{Pair Associative Container}{PairAssociativeContainer}.
\end{itemize}


\algosubsection{Notation}
$F$ is the set of facets of \variable{C}.
\\
$I$  the set of  \glossarylink{interior facets}{interior facet}
of  \variable{C}.
\\$B$ the \glossarylink{boundary facets}{boundary facet} of  \variable{C}
(such that $I \cup B = F$).
\\
$D$ is the domain of the map \variable{F} before the call.
\\
$D^\prime$ is the domain of the map \variable{F} after the call.

\algosubsection{Preconditions}
The grid underlying \variable{C} is a 
subgrid of a \glossarylink{manifold-with-boundary complex}{manifold-with-boundary}.
\\
For versions [2] and [3],
the domain of the map 
\variable{F} may already contain facets,
but these must be  boundary facets:
\[
   D \cap I = \emptyset
\]

\algosubsection{Postconditions}
All pairs of cells sharing an interior facets $\in I$ have been found,
as well as all cells having a Facet such that the corresponding 
vertex set has been in $D$.

After the call, 
\variable{F} 
contains exactly the boundary facets,
except those that have been in it before (versions [2], [3]).
In version [1], the value of \variable{Nb} is set to an invalid cell handle
for boundary facets (that is, facets which have been visited by only
facet-on-cell iterator).


\[
  D^\prime = (D \setminus B) \cup (B \setminus D)
\]
It follows that  $D^\prime \cap I = \emptyset$.

\algosubsection{Complexity}
Expected time $O(F)$.

\algosubsection{Example}
\begin{example}
#include "Grids/Algorithms/cell-neighbor-search.h"

\gralclasslink{Triang2D}{triang2d} t;
typedef grid_types<Triang2D> gt;
...
\xlink{hash_map}{\STLPATH{hash_map.html}}<gt::FacetOnCellIterator> , gt::cell_handle> Nbs;
CalculateNeighborCells(nbf,T);

for(gt::CellIterator c(t); ! c.IsDone(); ++c) {
  for(gt::FacetOnCellIterator fc(c); ! fc.IsDone(); ++fc)
    out << nbf[fc] << ' ';
  out << '\\\\n';
}
\end{example}
The mapping from \code{gt::FacetOnCellIterator} to \code{gt::cell\_handle}
can of course be done much more efficiently than with a hash table, see
the example in  \gralfilelink{test-triang2d-construct}{C}{triang2d}.

\algosubsection{Known Uses}

Useqd in test file \gralfilelink{test-triang2d-construct}{C}{triang2d}
for \gralclasslink{Triang2D}{triang2d}.

\algosubsection{Notes}

\algosubsection{See also}

