\xname{Grid-With-Boundary}
\begin{Label}{Grid-With-Boundary}
 \conceptsection{Grid-With-Boundary Concept}
\end{Label}

\conceptsubsection{Description}
A \concept{Grid-With-Boundary}  allows to query 
whether a facet is a \Glossarylink{boundary facet}.
Moreover, it defines a special cell handle for an `outside cell', 
and allows to query whether a cell is inside the grid.

\conceptsubsection{Refinement of}
\conceptlink{Grid}{Grid}
\\
\conceptlink{Cell Grid Range}{VertexGridRange}
\\
\conceptlink{Facet Grid Range}{VertexGridRange}
\\
The cell type of a  \concept{Grid-With-Boundary} is a model
of \conceptlink{Grid Cell}{GridCell} and of 
\conceptlink{Facet Grid Range}{VertexGridRange}.


\conceptsubsection{Notation}
\type{G} is a type which is a model of \concept{Grid-With-Boundary}\\
\variable{g} is an object of type \type{G}\\
\variable{c} is an object of type \type{G::Cell} \\
\variable{h} is an object of type \type{G::cell\_handle}\\
\variable{f} is an object of type \type{G::Facet}\\
\variable{fc} is an object of type \type{G::FacetOnCellIterator}

\conceptsubsection{Definitions}

\conceptsubsection{Associated types}
No new ones, but the requirements on \code{G::Cell}
are strengthened.

\begin{tabular}{ccc} 
  \T \\   \hline
  {\bf  Name  } & {\bf  Expression  } & {\bf  Description  } \\ 
  \hline
  cell type  & \code{G::Cell} &
  the cell element type of \code{G}
  \par model of \conceptlink{Grid Cell}{GridCell}
  and of \conceptlink{Facet Grid Range}{VertexGridRange} 
  \T \\  \hline  \\
\end{tabular}

\conceptsubsection{Valid Expressions}

\noindent
\begin{tabular}{llll} 
  \T \\  \hline
  \bf  Name  &\bf  Expression  &\bf  Type requirements  & \bf  return type  \\ 
  \hline
  inside test &
  \code{g.IsInside(c)} &
   ~ & 
  \type{bool} \\
  inside test &
  \code{g.IsInside(h)} &
  ~  & 
  \type{bool} \\
  boundary test &
  \code{g.IsOnBoundary(fc)} &
  ~ & 
  \type{bool} \\
  boundary test&
  \code{g.IsOnBoundary(f)} &
  ~  & 
  \type{bool} \\
  outer cell handle &
  \code{g.outer\_cell\_handle()} &
  ~ &
  \type{G::cell\_handle}
  \T \\  \hline  \\
\end{tabular}

\conceptsubsection{Expression semantics}

\noindent
\begin{tabularx}{14cm}{RRRRR} 
  \T \\  \hline
  \bf  Name       &
  \bf  Expression &
  \bf  Precondition&
  \bf   Semantics &
  \bf   Postcondition
  \\
  \hline
  inside test &
  \code{g.IsInside(c);} &
  \variable{c} is \footlink{bound}{bound}   to \variable{g}
  \par
   \variable{c} is  \footlink{valid}{valid} 
   or \code{c.handle() == g.outer\_cell\_handle()} &
   Test if \variable{c} is inside the grid &
   true if \variable{c} is  \link{valid}{valid} for \variable{g}.
   \par
   false if  \code{c.handle() == g.outer\_cell\_handle()}
   \\
   inside test&
  \code{g.IsInside(h);} &
   \variable{h} is  \footlink{valid}{valid} 
   or \code{h == g.outer\_cell\_handle()} &
   Test if \variable{h} is handle of a cell inside the grid &
   true if \variable{h} is  \link{valid}{valid} for \variable{g}.
   \par
   false if  \code{h == g.outer\_cell\_handle()} 
   \\
   boundary test &
   \code{g.IsOnBoundary(f)} &
   \variable{f} is \link{valid}{valid} &
   Returns \code{true} if \variable{f}
   is a \Glossarylink{boundary facet} of \variable{g} &
   ~ 
   \\
   boundary test &
   \code{g.IsOnBoundary(fc)} &
   \variable{fc} is \link{valid}{valid} &
   equivalent to \code{g.IsOnBoundary(*fc)}.& 
   ~ 
   \\
   outer cell handle &
  \code{h = g.outer\_cell\_handle();} &
   ~ &
  return a handle denoting an `outside' cell &
  \code{g.IsInside(h)} is false 
  \T \\  \hline
\end{tabularx}

\conceptsubsection{Invariants}

\noindent
\begin{tabularx}{14cm}{RR}
  \T \\  \hline
  & \code{g.IsInside(g.outer\_cell\_handle()) == false}\\
  & \code{g.IsInside(c) == g.IsInside(c.handle())} \\
  & \code{g.IsOnBoundary(fc) == ! g.IsInside(fc.OtherCell())}\noteref{note-gwb-othercell} \\
  & \code{g.IsOnBoundary(f) == (!g.IsInside(f.TheCell()) || !g.IsInside(f.OtherCell()))} \\
  & \code{g.IsInside()} evaluates to \code{true} for all
  cells in the range [\code{g.FirstCell()},\code{g.EndCell()}).
  \T \\  \hline
\end{tabularx}

\conceptsubsection{Refinements}

\conceptsubsection{Models}
\sectionlink{\type{Complex2D}}{Complex2D}   

\conceptsubsection{Notes}
\begin{enumerate}
\item \notelabel{note-gwb-outer-non-unique}
It does however not follow that if 
\code{! g.IsInside(h)}, then
{\code h == g.outer\_cell\_handle()}, 
because it is possible that there are many different handles
denoting an outer cell. For example, when we consider Cartesian grids,
cells with an logical index $(i,j)$ outside the grid range would 
satisfy \code{! g.IsInside(c)}
but do not necessarily possess a handle equal to
{\code g.outer\_cell\_handle()}.

\item  \notelabel{note-gwb-othercell}
It is possible to create a cell with the `outer' cell handle:
\begin{example}
 Cell c = g.cell(g.outer_cell_handle())
\end{example}
is possible, as is 
\begin{example}
Cell c = fc.OtherCell()
\end{example} 
when \code{g.IsOnBoundary(fc)} is true.
However, the only operation guaranteed to work with \variable{c} is
the query \code{g.IsInside(c)} which evaluates to \code{false}.
\end{enumerate}
\conceptsubsection{See also}


