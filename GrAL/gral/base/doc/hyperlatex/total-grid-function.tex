  
\xname{TotalGridFunction}
\begin{Label}{TotalGridFunction}
  \conceptsection{Total Grid Function Concept}
\end{Label}

\conceptsubsection{Description}

The  {\em Total Grid Function\/} concept refines the 
\sectionlink{\concept{ Container Grid Function}}{ContainerGridFunction} concept.
A total grid function reserves storage to hold a value for each element in
its range.

\conceptsubsection{Refinement of}
\sectionlink{\concept{ Container Grid Function}}{ContainerGridFunction}

\conceptsubsection{Notation}
{\tt F} is a type which is a model of Total  Grid  Function 
\\
{\tt f} is an object of type  {\tt F}
\\
{\tt G} is shorthand for  {\tt F::grid\_type}
\\
{\tt g} is an object of type  {\tt G}.

\begin{ifhtml}
\conceptsubsection{Associated types}
None, exept those defined in
\sectionlink{\concept{ Container Grid Function}}{ContainerGridFunction}

\conceptsubsection{Valid Expressions}
None, exept those defined in
\sectionlink{\concept{ Container Grid Function}}{ContainerGridFunction}
\end{ifhtml}

\conceptsubsection{Expression semantics}
\begin{tabularx}{15cm}{RRRRR}
  \hline    
  \bf  Name     &
  \bf  Expression &
  \bf  Precondition&
  \bf  Semantics &
  \bf  Postcondition
  \\ 
  \hline
  construction from grid & 
  {\tt F f(g);} &
  ~ &
  construct and bind {\tt f}  to {\tt g},
  allocate memory for {\tt f.size()} values. &
  {\tt f} is \link{bound}[\footnote{see \Ref}]{bound} to {\tt g} 
  \par write access is allowed 
  \par read access is undefined
  \par {\tt f.size()} is equal to the cardinality of the 
  \link{range}[\footnote{see \Ref}]{range}  of {\tt f}
  \\ 
  construction and initialization & 
  {\tt F f(g,t);} &
  ~ &
  construct and bind {\tt f}  to {\tt g}, 
  \par allocate memory for {\tt f.size()} values,
  \par initialize all values to {\tt t} 
  &
  {\tt f} is bound to {\tt g} 
  \par write access is allowed 
  \par {\tt f(e)} is equal to {\tt t} for all elements {\tt e}
  in the range of {\tt f}.
  \par {\tt f.size()} is equal to the cardinality of the 
  \link{range}{range} of {\tt f}
  \\ 
  Binding to grid &
  {\tt f.set\_grid(g);} &
  f is unbound &
  bind {\tt f}  to {\tt g},
  \par allocate memory for {\tt f.size()} values. &
  {\tt f} is bound to {\tt g} 
  \par write access is allowed, 
  \par read access is undefined
  \par {\tt f.size()} is equal to the cardinality of the 
  \link{range}{range} of {\tt f}
  \\ 
  \hline
  \\
\end{tabularx}

\conceptsubsection{Complexity Guarantees}
Default construction takes constant time.
\\
Construction from grid and construction with initalization both
take time at  O({\tt f.size()}), that is, the number of 
elements of type {\tt F::element\_type} of {\tt g}.



\conceptsubsection{Models}
\sectionlink{{\tt grid\_function\_vector<E,T>}}{grid-function-vector}
defined in
\xlink{{\tt grid-function-vector.h}}{\NMWRINC/Grids/grid-function-vector.h}

Total grid functions for the
\sectionlink{{\tt Complex2D}}{Complex2D}
concrete grid, defined in
\xlink{{\tt Complex2D/grid-functions.h}}{\NMWRINC/Grids/Complex2D/grid-functions.h}.

For {\tt E = \sectionlink{\concept{Complex}2D::Vertex}{Complex2D::Vertex}} 
and {\tt E = Complex2D::Cell},
the total grid functions are derived from
\sectionlink{{\tt grid\_function\_vector<E,T>}}{grid-function-vector};
and for  {\tt E = Complex2D::Edge}, 
the total grid function is derived from 
\sectionlink{{\tt grid\_function\_hash<E,T>}}{grid-function-hash}.
The reason  is taht edges are not stored in the 
\sectionlink{{\tt Complex2D}}{Complex2D}
data structure, and hence there is no consecutive index available for type  
{\tt Complex2D::Edge}.


\W\conceptsubsection{Notes}
    

\conceptsubsection{See also}
\sectionlink{\concept{ Grid Element Function }}{GridElementFunction} ~
\sectionlink{\concept{ Grid  Function }}{GridFunction} ~
\sectionlink{\concept{ Mutable Grid  Function }}{MutableGridFunction} ~
\sectionlink{\concept{ Container Grid  Function }}{ContainerGridFunction} ~
\sectionlink{\concept{ Partial Grid Function }}{PartialGridFunction} ~

  

