\xname{VolumeGridGeometry}
\begin{Label}{VolumeGridGeometry}
 \conceptsection{Volume Grid Geometry Concept}
\end{Label}

\conceptsubsection{Description}
 The \concept{Volume Grid Geometry} assigns a geometric entity
to every combinatorial element of a grid. 
Moreover, the it offers methods to calculate the 
volumes of these geometric entities, 
which means volume in the combinatorial dimension of the 
entities, for example, length of edges or area of faces.

\conceptsubsection{Refinement of}
\conceptlink{Vertex Grid  Geometry}{VertexGridGeometry}

\conceptsubsection{Notation}
\type{Geo} is a type which is a model of \concept{Vertex Grid Geometry} \\
{\tt g} is an object of type \type{Geo}\\
{\tt x} is an object of type \type{Geo::real\_type} \\
{\tt q} is an object of {\tt Geo::coord\_type}\\
{\tt s} is an object of type \type{Geo::segment\_type} \\
{\tt p} is an object of type \type{Geo::polygon\_type}\\
{\tt h} is an object of type \type{Geo::polyhedron\_type}\\
{\tt e} is an objects of type \type{Geo::grid\_type::Edge},
 which is a model of \conceptlink{Grid Edge}{GridEdge}.\\
{\tt f}  is an objects of type \type{Geo::grid\_type::Face},
which is a model of \conceptlink{Grid Face}{GridFace}.\\
{\tt c}  is an objects of type \type{Geo::grid\_type::Cell},
which is a model of \conceptlink{Grid Cell}{GridCell}.\\

\W\conceptsubsection{Definitions}

\conceptsubsection{Associated types}
\begin{tabularx}{14cm}{llR} 
  \\ \hline
  \bf  Name  & \bf  Expression  &\bf  Description   \\
  \hline
   point type &
   {\tt Geo::coord\_type}&
   the geometric point type, 
   representation of the elements of the topological 
   space where the geometry lives
   \par model
   of STL \Stllink{Assignable}.
   \\
   real number type & 
   \type{real\_type} &
   representation of a real number,
   model of   STL \Stllink{Assignable} 
   \\
   segment type &
   \type{segment\_type} & 
   geometric type corressponding to edges,
   \par model of   STL \Stllink{Assignable} 
   \\
   polygon type &
   \type{polygon\_type} &
   geometric type corresponding to faces 
   (if the grid type has combinatorial dimension $\geq 2$.)
   \par model of   STL \Stllink{Assignable} 
   \\
   polyhedron type &
   \type{polyhedron\_type} &
   geometric type corresponding to cells 
   (if the grid type has combinatorial dimension $3$.)
   \par model of   STL \Stllink{Assignable} 
   \\
   \hline
 \end{tabularx}

\conceptsubsection{Valid Expressions}
 
\begin{tabularx}{14cm}{RRRR}
  \\ \hline
  \bf  Name  &\bf  Expression  &\bf  Type requirements  & \bf  return type  
  \\
  \hline
   $1$-dimensional geometric entity &
   \pcode{s = g.segment(e)} &
   & \type{Geo::segment\_type}
   \\
   $2$-dimensional geometric entity &
   \pcode{p = g.polygon(f)} &
   & \type{Geo::polygon\_type}
   \\
   $3$-dimensional geometric entity &
   \pcode{h = g.polyhedron(c)} &
   & \type{Geo::polyhedron\_type}
   \\
   $1$-dimensional volume &
   \pcode{x = g.volume(e);} &
    & real\_type
    \\
   $1$-dimensional volume &
   \pcode{x = g.length(e);} &
    & real\_type
    \\
   $2$-dimensional volume &
   \pcode{x = g.volume(f);} &
    & real\_type
    \\
   $2$-dimensional volume &
   \pcode{x = g.area(f);} &
    & real\_type
    \\
   $3$-dimensional volume &
   \pcode{x = g.volume(c);} &
    & \type{real\_type}
    \\
  \hline
\end{tabularx}

\conceptsubsection{Expression semantics}

The general requirement here is that the mapping to geometric entities
is faithful with respect to the combinatorial structure,
 that is, the poset defined by the inclusion relation of the geometric elements
is identical or a refinement of the poset of the grid:
The relative boundary of a geometric image of a grid element is identical of
the union of the geometric images of its lower-dimensional incidents.
Note, however, that self-intersecting immersions are not excluded.

 \begin{tabularx}{15cm}{RRRRR} 
   \\ 
   \hline 
   \bf  Name    &
   \bf  Expression &
   \bf  Precondition&
   \bf  Semantics &
   \bf  Postcondition
   \\ 
   \hline
   $1$-dimensional geometric entity &
   \pcode{s = g.segment(e)} &
   \pcode{e} is \footlink{valid}{valid} &
   get the segment corresponding to \pcode{e} &
    \pcode{s == g.segment(e)} 
    \par \pcode{start(s) == g.coord(e.V1())}
    \par \pcode{end(s) == g.coord(e.V2())}
    \\
    $2$-dimensional geometric entity &
    \pcode{p = g.polygon(f)} &
    \pcode{f} is \link{valid}{valid} &
    get the polygon corresponding to \pcode{f} &
    $\texorhtml{\partial}{\mbox{boundary of}}
      \mbox{\tt p} = \bigcup_{\mbox{\tt e} \prec \mbox{\tt f}} \mbox{\tt g.segment(e)}$
      \\
      \hline
\end{tabularx}

\conceptsubsection{Invariants}
\conceptsubsection{Refinements}

\conceptsubsection{Models}
\conceptsubsection{Notes}
\conceptsubsection{See also}


